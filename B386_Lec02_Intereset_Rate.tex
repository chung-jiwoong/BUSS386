\documentclass[10pt]{beamer}
\usepackage{etex}

%\documentclass[handout]{beamer}
\usepackage{amsmath,amsfonts,amssymb}
\usepackage{natbib}
\usepackage{tabularx,color,colortbl}
\usepackage{graphicx}
\usepackage{cancel}
\usepackage{multirow}
\usepackage{multicol}
\usepackage{comment}\excludecomment{hide}
\usepackage{subfigure}
\usepackage{array}
\usepackage[singlelinecheck=off]{caption}
\usepackage{booktabs}
\usepackage{fixltx2e}
\usepackage[flushleft]{threeparttable}
\usepackage[font=footnotesize,labelfont=bf]{caption}
\usepackage{lscape}
\usepackage{slashbox}
\usepackage{rotating}
\usepackage{epstopdf}
\usepackage{tikz}
\usepackage{xypic}


\usepackage{tikz}
\usetikzlibrary{patterns,decorations.pathreplacing}

%\usepackage{pgfpages}
%\setbeameroption{show notes}
\setbeameroption{hide notes}
%\setbeameroption{show notes on second screen}

%\usepackage{pgfpages}
%\pgfpagesuselayout{4 on 1}[letter,border shrink=5mm]

%\usepackage{handoutWithNotes}
%\pgfpagesuselayout{3 on 1 with notes}[a4paper,border shrink=5mm]


\newcolumntype{N}{>{\raggedleft\arraybackslash}X}
\newcolumntype{L}{>{\raggedright\arraybackslash}X}
\newcolumntype{R}{>{\raggedleft\arraybackslash}X}%
\newcommand{\Ind}{\,\rotatebox[origin=c]{90}{\ensuremath{\models}}\,} %prob independence
\newcommand{\tab}{\hspace*{2em}}
\newenvironment{proenv}{\only{\setbeamercolor{local structure}{fg=white}}}{}
\newenvironment{conenv}{\only{\setbeamercolor{local structure}{fg=blue}}}{}

\setbeamertemplate{itemize itemsep}[15pt]


%\usetheme{Ilmenau}
\usetheme{boadilla}
\usecolortheme{dolphin}
\useinnertheme{circles}

\usenavigationsymbolstemplate{}
\setbeamertemplate{footline}[frame number]


%\setbeamerfont{institute}{size=\footnotesize}
%\setbeamerfont{author}{size=\small}
%\setbeamerfont{date}{size=\small}



\title{Interest Rates \\
 \vspace{5pt} {\footnotesize BUSS386. Futures and Options}}
\author{Prof. Ji-Woong Chung}
\institute{}
\date{}

\begin{document}

\frame{\titlepage}
\graphicspath{{figures//}}


\begin{frame}
\frametitle{Lecture Outline}
	\begin{itemize}\itemsep15pt
		\item Interest Rates \\ \vspace{5pt}
		\begin{itemize}\itemsep10pt
			\item Types
			\item Measurement
			\item Zero/Spot/Forward rates
			\item Forward Rate Agreements
			\item Duration and convexity
		\end{itemize}
		\item Reading: \S Ch. 4
	
	\end{itemize}
\end{frame}


\section{Interest Rates}


\begin{frame}
	\begin{center}
		{\Large Interest Rates: Types and Measurement}
	\end{center}
\end{frame}



\begin{frame}
	\frametitle{Interest Rates}
	\begin{itemize} \itemsep15pt
		\item For valuation of derivatives (and any financial asset), we need interest rates.
		\item There are different types and measurements of the interest rates.
		\item In general, interest rates are higher for riskier investments.\vspace{5pt}
		\begin{itemize}
			\item \textbf{Risk-free interest rate} is the rate for investment having zero risk.
		\end{itemize}		
	\end{itemize}
\end{frame}


\begin{frame}
	\frametitle{Types of Interest Rates}
	\begin{itemize} \itemsep10pt
		\item Treasury Rates \vspace{5pt}
		\begin{itemize}\itemsep5pt
			\item The rate investors earn when they invest in government (treasury) bonds.
			\item Government can always pay back the principal and interest (really?), so investing in the government bonds is risk-free ($\Rightarrow$ risk-free rate).			
		\end{itemize}
		\item Overnight Rates \vspace{5pt}
		\begin{itemize}\itemsep5pt
			\item Banks are required to maintain a certain amount of cash, known as a reserve, with the
			central bank.
			\item Borrowing and lending overnight between surplus and deficit banks.
			\item Federal funds rate (U.S.), SONIA (U.K.), ESTER (Euro), SARON (Swiss), TONAR (Japan)
		\end{itemize}

		\item Repo Rates \vspace{5pt}
		\begin{itemize}\itemsep5pt
			\item Unlike the overnight rate, repo rates are secured borrowing rates
			\item SOFR (U.S.), KOFR (Korea)
		\end{itemize}
		
	\end{itemize}
\end{frame}


\begin{frame}
	\frametitle{Reference Rates}
	\begin{itemize} \itemsep10pt
		\item Reference interest rates are important in financial markets.
		
		\item LIBOR (London Interbank Offered Rate)\vspace{5pt}
		\begin{itemize}\itemsep5pt
			\item The rate at which banks ``could" obtain loan from other banks
			\item A bank should be creditworthy (e.g. with the credit rating of AA or higher) to be able to borrow at LIBOR.
			\item This is an unsecured loan, so the borrowing bank may default, in theory.
			\item However, the chance of default is very small, because banks participating in the LIBOR market have high credit ratings $\Rightarrow$ (almost) risk-free rate
		\end{itemize}
	\end{itemize}
\end{frame}




\begin{frame}
	\frametitle{LIBOR Phase-out}
		\begin{itemize}\itemsep5pt
			\item The LIBOR Scandal in 2012
			\item Regulators plan to phase out LIBOR by the end of 2021 and replace it with rates based on \textbf{transactions} observed in the overnight market.
			\item New reference rates \tiny{(* based on repo rate)}:
			\begin{itemize}\itemsep5pt
				\item US dollar: SOFR (secured overnight funding rate)$^*$
				\item Korea: KOFR (Korea overnight financing repo rate)$^*$
				\item GBP: SONIA (sterling overnight index average)
				\item EU: ESTER (euro short-term rate)
				\item Switzerland: SARON (Swiss average overnight rate)
				\item Japan: TONAR (Tokyo average overnight rate) 
			\end{itemize}
			
		\end{itemize}
\end{frame}



\begin{frame}
	\frametitle{TED Spread}
	\begin{center}
		Treasury Rates v.s. LIBOR\\ \vspace{10pt}
		\includegraphics[scale=0.35]{Fig01}
		{\scriptsize The TED spread=3-month LIBOR $-$ 3-month T-bill rate in USD is always positive. Now discontinued. }
	\end{center}
	New reference rates are risk-free, so do not create volatile spreads.
\end{frame}



\begin{frame}
	\frametitle{Measuring Interest Rates}
	\begin{itemize} \itemsep15pt
		\item Interest rates are quoted as an \textbf{annual} rate.
		\item However, the actual frequency at which interests are earned (or compounded) does not have to be annual  \vspace{5pt}
		\begin{itemize}\itemsep5pt
			\item semiannual compounding: interests are earned in every six months.
			\item quarterly compounding: interests are earned in every quarter.
			\item monthly compounding: interests are earned in every month.
			\item $\vdots$
		\end{itemize}
		\item Let $r$ denote the annual rate and $m$ denote the number of compounding periods in a year. If you invest \$1 for one year, then future value is
		$$
			\$1\times\left( 1 + \frac{r}{m} \right)^m
		$$
	\end{itemize}
\end{frame}

\begin{frame}
	\frametitle{Measuring Interest Rates}
	\begin{itemize}\itemsep20pt
		\item \textbf{Continuous compounding} is an extreme case where interests are earned at every instant.
		\item With the annual rate $r$, the future value of \$1 after one year with continuous compounding is
		$$
			\$1\times e^{r}
		$$
		\item The future value after $T$ years is
		$$
			\$1\times e^{rT}
		$$
	\end{itemize}
\end{frame}



\begin{frame}
	\frametitle{Conversion of Interest Rates}
	\begin{itemize}\itemsep20pt
		\item In some cases, we need to convert the interest rate with continuous compounding to the rate with different compounding frequency (or vice versa).
		\item Let $r_c$ denote the rate with continuous compounding. Also, let $r_m$ denote the rate with compounding $m$ times per year. Then,
		$$
			e^{r_c} =\left( 1+ \frac{r_m}{m} \right)^m
		$$
		\item[Ex.] An interest rate is quoted as 10\% per annum with semiannual compounding. What is the equivalent rate with continuous compounding? Higher or lower?
		\item[NB] Given $P_{t+1}$ and $P_t$,  $r_c =\ln\frac{P_{t+1}}{P_t} $. That is, $P_{t+1}=e^{r_C}P_t$.
	\end{itemize}
\end{frame}

% Why to use continuous compounding here?:

%It's a good question. We don't *need* to, really. But it is convenient for us, esp. in derivatives pricing, due to its properties; e.g., it is often easier to differentiate with the easy properties of the LN() and EXP() functions. Linda Allen (assigned Quant chapter) alludes to this when she notes continuous returns are "time consistent." That is, they can be easily added. For example, 2% return in year 1, then 3% in year 2 = 5% over 2 years because EXP(2%)*EXP(3%)=EXP(5%). But, if those were instead annual compound frequency, (1.02)(1.03)>1.05. It goes a little further, as the product of the continuous returns, if normal, is also normal. But it is not the case for non-continous.

%That said, for the FRM, we still want to be able to translate from continuous to the various compound frequencies. Hull does perform the forward/cost of carry models in continuous, as you suggest, but also to your point, the forward contract can alternatively be done in discrete compounding. Nothing wrong with that. Specifically, Hull computes implied forward interest rates with continuous but the FRM assigned on this topic is Tuckman and he uses semi-annual compounding b/c he does that for all bonds. So, our ideal (IMO) is to be able to switch from one to another, seeing them as but different frequencies (where continuous is the convergence).

%https://www.bionicturtle.com/forum/threads/why-to-use-continuous-compounding.612/



\begin{frame}
	\frametitle{Zero Rates}
	\begin{itemize}\itemsep20pt
		\item Consider a risk-free zero-coupon bond where all principal and interests are paid at the end of $n$ years.		
		\item The interest rate on such investment is called $n$-year zero-coupon interest rate (in short, we also call  $n$-year zero rate or $n$-year spot rate).		
		
		\item[Ex.] 5-year zero-coupon bond with the principal of \$1,000 is priced at \$890. What is the 5-year zero rate with continuous compounding?\\ \vspace{5pt}
		$\Rightarrow$ Solving for $r$ such that $890\times e^{r5}=1000$, $r$=2.33\%.				
	\end{itemize}
\end{frame}




	

\begin{frame}
	\frametitle{Zero Rates - Facts I}
	\begin{itemize}\itemsep5pt
		\item In the market, the zero rates are different depending on investment horizon (maturity).\\ \vspace{10pt}		
		\begin{center}
		{\scriptsize US-treasury yield on 25 November 2016}\\
			\includegraphics[scale=0.4]{Fig02} \\
		{\tiny [source: https://www.treasury.gov/resource-center/data-chart-center/interest-rates/Pages/default.aspx] }
		\end{center}
		\item The relation between the interest rates and maturities is called \textbf{term structure} of interest rates.
	\end{itemize}
\end{frame}

\begin{frame}
	\frametitle{Zero Rates - Facts II}
	\begin{itemize} \itemsep15pt
		\item The interest rate for a certain maturity is time-varying. \\ \vspace{10pt}
		\begin{center}
		{\scriptsize US-treasury yield for 2-year maturity from 1976 to 2020}\\
			\includegraphics[scale=0.25]{Fig03} \\
		{\tiny [source: https://fred.stlouisfed.org] }
		\end{center}
	\end{itemize}
\end{frame}






\begin{frame}
	\frametitle{Interest Rates - Notation}
	\begin{itemize}
		\item Given the fact that interest rates differ depending on the maturity, we use the following notation:
		$$
			r_0(t_1,t_2)
		$$
		is the interest rate from $t_1$ to $t_2$ (with continuous compounding).\\ \vspace{10pt}
	\end{itemize}
	
\begin{center}	
\xymatrix@R=2pt{
~~T=0~~ & ~~1~~ & ~~2~~ &  ~~3 \text{ year}~~   \\
\ar@{-}[rrr] \ar@{-}[rrr]& & &  \\
\\
\ar@{-}[uu]  &  \ar@{-}[uu] &  \ar@{-}[uu] &  \ar@{-}[uu]  \\
\ar[r]_{r_0(0,1)} & & & \\
\\
\\
\ar@{-}[rr] & \ar[r]_{r_0(0,2)} & & \\
\\
\\
\ar@{-}[rr] & \ar@{-}[rr] & \ar[r]_{r_0(0,3)}  & \\
}
\end{center}
\end{frame}


\begin{frame}
	\frametitle{Bond Pricing}
	\begin{itemize}\itemsep10pt
		\item A 2-year bond with a principal of \$100 provides coupons at the rate of 6\% per annum semiannually. The theoretical price of the bond is
		$$
		3e^{-0.05(0.5)} + 3e^{-0.058(1.0)} + 3e^{-0.064(1.5)} + 103e^{-0.068(2.0)} = 98.39
		$$
		\begin{center}
			\includegraphics[scale=0.5]{Tab4.2} 
		\end{center}
		\item A \underline{yield-to-maturity} is the single discount rate that gives a bond price equal to its market price ($y=6.76\%$)
		$$
		3e^{-y(0.5)} + 3e^{-y(1.0)} + 3e^{-y(1.5)} + 103e^{-y(2.0)} = 98.39
		$$
	\end{itemize}
\end{frame}


\begin{frame}
	\frametitle{Bond Pricing}
	\begin{itemize}\itemsep10pt
		\item The par yield for a certain bond maturity is the coupon rate that makes the bond price equal its par value. 
		\item Suppose that the coupon on a 2-year bond is $c$ per annum. Using the zero rates in Table 4.2, the
		value of the bond is equal to its par value of 100 when $c=6.87\%$.
		$$
		\frac{c}{2}e^{-0.05(0.5)} + \frac{c}{2}e^{-0.058(1.0)} + \frac{c}{2}e^{-0.064(1.5)} + \left(100+\frac{c}{2}\right)e^{-0.068(2.0)} = 100
		$$
		
		\item Useful for benchmarking and pricing.
	\end{itemize}
\end{frame}



\begin{frame}
	\frametitle{Forward Rates}
	\begin{itemize}\itemsep15pt
		\item[e.g.] Consider the following term-structure of interest rates: \\ \vspace{5pt}
		\begin{center}
			\begin{tabular}{cc}
			\hline
			Maturity (years) & Interest rate (\%) \\
			\hline
			1 & 3.0 \\
			2 & 4.0 \\
			\hline
			\end{tabular}
		\end{center}
		\item We can find the \textbf{implicit} rate that can be earned from year 1 to year 2. Let $r$ denote the rate. Then,
		$$
			e^{0.04\times 2} = e^{0.03}e^r
		$$ 	
		$r$ = 5.00\%.
		\item We call $r$ the \textbf{(implied) forward rate}.
	\end{itemize}		
\end{frame}


\begin{frame}
	\frametitle{Forward Rates - Borrowing/Lending}
	\begin{itemize}\itemsep20pt
		\item We can lock in interests for future borrowing/lending at this forward rate.
		\item Suppose that we want to borrow \$1 one year from now and repay in year 2. Also, we want to fix now the interest rate for this borrowing.
		
		\item If we fix at the forward rate, this borrowing will have cash flows as follows: 		
		\begin{center}
				\begin{tabular}{cccc}
				\hline
				Action & year 0 & year 1 & year 2 \\
				\hline	
				\begin{tabular}{@{}c@{}}borrowing \\ at forward rate\end{tabular} & 0 & 1 & $-e^{0.05\times 1}$ \\		
				\hline
				\end{tabular}
		\end{center}
	\end{itemize}
\end{frame}


\begin{frame}
	\frametitle{Forward Rates - Borrowing/Lending}
	\begin{itemize}\itemsep15pt
		\item We may not have a financial instrument where we can borrow or lend at the forward rate in the markets.
		\item Then, we can construct this by combining two zero coupon bonds. In this construction, we can choose the face-value of zero coupon bonds as we like.\\  				
		\begin{center}
			\begin{tabular}{cccc}
			\hline
			Action & year 0 & year 1 & year 2 \\		\hline	
			\begin{tabular}{@{}c@{}}buy 1-yr bond \\ (lend)	\end{tabular}  & $-e^{-0.03\times 1}$	& 1	& 0\\		
			\begin{tabular}{@{}c@{}}sell 2-yr bond \\ (borrow)	\end{tabular}   & $e^{-0.03\times 1}$ & 0 & $-e^{-0.03\times 1}e^{0.04\times 2}$ \\		\hline
			net & 0 & 1 & $-e^{0.05\times 1}$ \\	\hline
			\end{tabular}
		\end{center}			
	\end{itemize}
\end{frame}




\begin{frame}
	\frametitle{Forward Rates - Notation}
	\begin{itemize} \itemsep20pt
		\item In the forward rate, the time of measuring interest rate is different from the starting time of investment.
		\item Thus, we generalize the notation for interest rate:
		$$
			r_t(t_1,t_2)
		$$
		is the interest rate from time $t_1$ to $t_2$, measured on date $t$.		
	\end{itemize}
\end{frame}


%\begin{frame}[t]
%	\frametitle{Forward Rates - General Result}
%	\begin{itemize} \itemsep15pt
%		\item Once we know the term structure of spot rates, we can find the forward rates between $t_1$ and $t_2$ as follows:
%		\begin{align*}
%			r_0(t_1,t_2) = \frac{r_0(0,t_2)t_2 - r_0(0,t_1)t_1}{t_2 - t_1}	
%		\end{align*}			
%	\end{itemize}
%\end{frame}


\begin{frame}[t]
	\frametitle{Forward Rates - Example}
	\begin{itemize} \itemsep15pt
		\item  Consider the following term-structure of interest rates:
		\begin{center}
			\begin{tabular}{cc}
			\hline
			Maturity (years) & Interest rate (\%) \\
			\hline
			1 & 3.0 \\
			2 & 4.0 \\
			3 & 4.6 \\
			\hline
			\end{tabular}
		\end{center}	
		\item[Q1.] What is the implied forward rate $r_0(2,3)$?		
	\end{itemize}
\end{frame}


\begin{frame}[t]
	\frametitle{Forward Rates - Borrowing/Lending - Example}
	\begin{itemize} \itemsep15pt
		\item[Q2.] An investor wants to lend \$100 in year 2 and receive in year 3. The investor wants to lock in the interest rate at $r_0(2,3)$. Construct this instrument using two zero bonds.
	\end{itemize}
\end{frame}




%\begin{frame}
%	\frametitle{Forward Rates - Example}
%	\begin{itemize}\itemsep15pt
%	 \item Revisit the previous example with the following term-structure:
%	\begin{center}
%			\begin{tabular}{cc}
%			\hline
%			Maturity (years) & Interest rate (\%) \\
%			\hline
%			1 & 3.0 \\
%			2 & 4.0 \\
%			3 & 4.6 \\
%			\hline
%			\end{tabular}
%		\end{center}
%	\item We found that $r_0(2,3) = 5.8\%$.
%	\item Suppose that no asset in the market provides the forward rate.
%	\item How can we construct this asset?
%	\end{itemize}
%\end{frame}
%
%\begin{frame}
%	\frametitle{Forward Rates - Example}
%	\begin{itemize} \itemsep15pt
%	\item If we had an asset providing the forward rate, then \textbf{cash flows} will look like:
%	\begin{center}
%				\begin{tabular}{cccc}
%				\hline
%				Action & year 0 & year 2 & year 3 \\
%				\hline		
%				invest in the asset  & & -1 & $e^{0.058\times 1}$ \\				
%				\hline
%				\end{tabular}
%		\end{center}		
%	\item We can generate the same cash flow by taking positions in 2-year and 3-year bonds:
%	\begin{center}
%				\begin{tabular}{cccc}
%				\hline
%				Action & year 0 & year 2 & year 3 \\
%				\hline		
%				buy 3-yr bond  & $-e^{-0.046\times 3}e^{0.058\times 1}$ &  & $e^{0.058\times 1}$ \\	
%				(lend) \\
%				sell 2-yr bond & $e^{-0.046\times 3}e^{0.058\times 1}$	& -1	& \\	
%				(borrow)		\\		
%				\hline
%				net & 0 & -1 & $e^{0.058\times 1}$ \\
%				\hline
%				\end{tabular}
%		\end{center}			
%	\end{itemize}		
%\end{frame}
%
%
%
%\begin{frame}[t]
%	\frametitle{Forward Rates - Example}
%	\begin{itemize}
%		\item[Q.] In the previous example, we find that $r_0(1,2) = 5\%$. If we had an asset providing the forward rate, the cash flows will look like:\\
%		
%		\begin{center}
%				\begin{tabular}{cccc}
%				\hline
%				Action & year 0 & year 1 & year 2 \\
%				\hline		
%				invest in the asset  & & -1 & $e^{0.05\times 1}$ \\				
%				\hline
%				\end{tabular} \\
%		\end{center}	
%		
%		How can we construct this asset by combining zero coupon bonds? \\ \vspace{60pt}
%	\end{itemize}
%
%\end{frame}


\begin{frame}
	\frametitle{Forward Rate Agreements}
	\begin{itemize} \itemsep15pt
		\item A forward rate agreement (FRA) is an agreement to exchange a predetermined fixed rate for a reference rate that will be observed in the market at a future time (the principal itself is not exchanged).
		\item Consider an agreement to exchange
		3\% for three-month LIBOR in two years with both rates being applied to a principal of \$100 million. Party A (B) would agree to pay (receive) LIBOR and receive (pay) the fixed rate of 3\%. Assume all rates are compounded quarterly. If three-month LIBOR proved to be 3.5\% in two years, Party B would receive
		$$
		\$100,000,000 \times (0.035 - 0.030) \times 0.25 = \$125,000
		$$
		
		\item As LIBOR is phased out, we can expect to see more FRAs based on floating rates such as three-month SOFR and three-month SONIA.
	\end{itemize}
\end{frame}



\begin{frame}
	\frametitle{Forward Rate Agreements}
	\begin{itemize} \itemsep15pt
		\item At initiation, an FRA rate is set so as to have zero value. That is, the FRA rate = forward rate.
		
		\item As time passes, the forward rate is likely to change. The FRA value deviates from zero.
		
		\begin{itemize}
			\item Example: forward SOFR rate for the period between time 1.5 years and time 2 years in the future is 5\% (with semiannual compounding) 
			\item Some time ago a company entered into an FRA where it will receive 5.8\% (with semiannual compounding) and pay SOFR on a principal of \$100 million for the
			period. 
			\item The 2-year (SOFR) risk-free rate is 4\% (with continuous compounding). 
			\item The value of the FRA is
			$$
			\$100,000,000 \times \frac{(0.058 - 0.050)}{2} \times e^{-0.04(2)} = \$369,200
			$$
		\end{itemize}
		
	\end{itemize}
\end{frame}





\section{Interest Rate Risk}


\begin{frame}
	\begin{center}
		{\Large Interest Rate Risk: Measurement and Management}
	\end{center}
\end{frame}




\begin{frame}
	\frametitle{Interest Rate Risk}
	\begin{itemize} \itemsep20pt
		\item Interest rate or yield volatility translates into price volatility for fixed income securities.
		
		\item Financial institutions like commercial and investment banks, and fixed
		income portfolio managers, are highly exposed to interest rate volatility
		
		\item These institutions manage interest rate risk in a variety of ways that include
		forward, future and swap contracts, and dynamic hedging strategies
	\end{itemize}
\end{frame}


\begin{comment}


\begin{frame}
	\frametitle{Yield to Maturity}
	
	\begin{itemize} 
		\item Yield to maturity is a convenient way to express price
		in terms of a single rate of interest (IRR).
		\item It is the single $y$ that solves (semi-annual coupon):
		$$
		\frac{c/2}{(1+r_{0.5}/2)^{1}} + \frac{c/2}{(1+r_1/2)^2} + ... + \frac{c/2+M}{(1+r_T/2)^{2T}}
		$$
		$$
		= \frac{c/2}{(1+y/2)^{1}} + \frac{c/2}{(1+y/2)^2} + ... + \frac{c/2+M}{(1+y/2)^{2T}}
		$$ \vspace{10pt}
		
		\item Realized return can be different from the YTM.
		\begin{itemize}
			\item Ex-post return is equal to its initial yield if all of the coupons are reinvested at the initial yield
			% and ii) if the yield at the end of the investment horizon is the same as the initial yield.
		\end{itemize}
		
	\end{itemize}
	
\end{frame}

\end{comment}




\begin{frame}
	\frametitle{Measurement: Duration and Convexity}
	
	\begin{itemize} \itemsep15pt
		\item A security’s price is function of its yield ``y'' and other factors: $ p(y; others)$:
		\begin{itemize}
			\item Other factors include maturity, coupon, embedded options, default risk, market
			conditions, etc.
		\end{itemize}
		
		\item Consider a bond with cash flows $K_{i}$ (coupon + par value) at time $t_i \in \{t_1, t_2, ..., t_N \}$.
		
		\begin{itemize}
			\item $P_0 = \sum_{i=1}^{N} K_i e^{-y_c t_i}$, when $y_c$ is quoted as continuously compounded.
			
			\item $P_0= \sum_{i=1}^{N} K_i (1+y_m/m)^{-mt_i} $ when $y_m$ is quoted as compounded $m$ times per year.
			
			% N = kT, where k=coupon frequency
			% When coupon frequency (k) = compounding frequency (m), $mt_i = i$
		\end{itemize}
		
		
		
		\item Duration measures are related to the first partial derivative: $\partial P / \partial y $
		
		\item Convexity measures are related to the second derivative: $ \partial^2 P / \partial y^2 $
		
	\end{itemize}
	
\end{frame}





\begin{frame}
	\frametitle{Duration: Basics}
	
	\begin{itemize} \itemsep15pt
		\item Duration measures the first order bond price sensitivity to interest rate changes.
		
		\begin{itemize}
			\item  Higher duration means higher price sensitivity to interest rate changes (more price volatility).
			\item More precisely, it is the elasticity of a bond price with respect to its yield
		\end{itemize}
		
		\item Macaulay duration: \% change in $P$ w.r.t \% change in $y$.\footnote{In 1938, Canadian economist Frederick R. Macaulay (``The Movement of Interest Rates, Bond Yields and Stock Prices in the United States Since 1856'') introduced this measure.}
		$$
		D_{Mac} = -\frac{dP/P}{dy/(1+y/m)} = -\frac{dP/P}{dy} (1+y/m)
		$$
		
		% Assume that m = k 
		
		\item Based on the promised cash flows. As such, it is only accurate for risk-free bonds with no embedded options.
		\item Duration is a property of a security or a portfolio at a point in time. It changes over time.
	\end{itemize}
	
\end{frame}


\begin{frame}
	\frametitle{Duration: Basics}
	
	\begin{itemize} \itemsep15pt
		\item The formula for Macaulay duration is: 
		
		$$
		D_{Mac} = \sum_{i=1}^{N} t_i \left( \frac{ K_i e^{-y_c t_i}}{P}  \right) (1+y_c/ m)
		$$
		
		$$
		D_{Mac} = \sum_{i=1}^{N} t_i \left( \frac{K_i \left( 1+ y_m/m \right)^{-mt_i}}{P} \right)  
		$$
		
		\item Macaulay duration is a weighted average of arrival time of
		cash flows, where the weights are the fraction of present value represented by that cash flow.
		
		\begin{enumerate}
			\item Macaulay duration of an option-free coupon bond is less than
			or equal to its time to maturity
			\item Macaulay duration of a zero coupon option-free bond is equal
			to its time to maturity
			\item The higher the coupon rate the shorter the duration
			\item As market yield increases, duration decreases
		\end{enumerate}
	\end{itemize}
	
\end{frame}




\begin{frame}
	\frametitle{Modified Duration}
	
	\begin{itemize} \itemsep15pt
		\item Modified duration: $D = D_{Mod} = D_{Mac}/(1+y/m)$
		
		\begin{itemize}
			\item $D_{Mod} =\sum_{i=1}^{N} t_i \left( \frac{ K_i e^{-y_c t_i}}{P}  \right)$ with $y_c$.
			
			\item $D_{Mod}= \sum_{i=1}^{N} t_i \left( \frac{ K_i \left( 1+y_m/m \right)^{-mt_i}  }{P} \right) \left( 1+ y_m/m \right)^{-1} $ with $y_m$.
		\end{itemize}
		
		
		
		
		\item Therefore, $D_{Mod} = - \frac{dP/P}{dy}$ 
		% This is the true interest risk of bond price. Macaulay is simply the weighteed time of cash flows!
				
		\item Rearranging it:  $\frac{dP}{P}  = -D_{Mod}dy$
		
		\begin{itemize} \itemsep10pt \vspace{10pt}
			\item The percentage price change is approximated by:
			(modified duration) times (the change in the yield).
			\item However, duration only gives an accurate
			estimate of price change for small yield changes.
			\item Due to the convexity of the price function.
		\end{itemize}
	\end{itemize}
	
\end{frame}



\begin{frame}
	\frametitle{Dollar Duration}
	
	\begin{itemize} \itemsep15pt
		\item Dollar duration, $D_d = D_{Mod} \times P = -\frac{dP/P}{dy} \times P = -\frac{dP}{dy}$
		
		\item Therefore, $dP = -D_d dy = -D_{Mod} P dy$
		
		\begin{itemize} \itemsep10pt \vspace{10pt}
			\item The dollar price change w.r.t. the change in yield.
			\item Dollar duration is useful in hedging strategies and for understanding risk of zero NPV portfolios (e.g. intereset rate swap)
		\end{itemize}
	\end{itemize}
	
\end{frame}



\begin{frame}
	\frametitle{Portfolio Duration}
	
	\begin{enumerate} \itemsep15pt
		\item The modified duration of a bond portfolio is the value weighted average modified duration of bonds in the
		portfolio.
		\item The dollar duration of a portfolio is the sum of the
		dollar durations of the bonds in the portfolio.
		
		\begin{itemize} \itemsep10pt \vspace{10pt}
			\item For a zero-value portfolio, only dollar duration is defined.
		\end{itemize}
	\end{enumerate}
	
\end{frame}




\begin{frame}
	\frametitle{Effective Duration}
	
	\begin{itemize} \itemsep15pt
		%\item These measures increase the accuracy of sensitivity estimates, which in turn makes risk assessment and hedging strategies more robust.
		
		\item Effective duration
		
		\begin{itemize} \itemsep10pt \vspace{10pt}
			\item An \textit{empirical} change in bond price w.r.t change in yield.
%			\item For risk-free securities, effective duration and modified duration are the same.
			\item For securities with uncertain cash flows due to optionality (like callable bonds, MBS with prepayment options, or securities with credit risk),
			effective duration can be significantly different from modified (or Macaulay) duration.
			\item The standard duration formulas are inaccurate when cash flows are highly uncertain. In such cases effective duration is the better measure.
			
			$$
			D_{eff} = -\frac{1}{P}\frac{dP}{dy} =-\frac{1}{P}\frac{P(+x \; bps) - P(-x \; bps)}{2x}
			$$
		\end{itemize}
	\end{itemize}
	
\end{frame}



%\begin{frame}
%	\frametitle{Generalized Duration}
%	
%	\begin{itemize} \itemsep15pt
%		\item Traditional duration measures price sensitivity to small changes in the general level of interest rates. But other factors also influence bond prices.
%		
%		\item Generalized measures of duration can be used to describe the total
%		\% change in bond price as the sum of the partial effects of multiple factors in a linear model:
%		$$
%		\frac{dP}{P} = \frac{1}{P} \left( \frac{\partial P}{\partial f_1} \Delta f_1 + \frac{\partial P}{\partial f_2} \Delta f_2  +  \dots + \frac{\partial P}{\partial f_n} \Delta f_n  \right)
%		$$
%		\begin{itemize} \itemsep10pt \vspace{10pt}
%			\item where $f_i$ is the $i$th factor that influences price.
%			\item $\frac{\partial P}{\partial f_i} \Delta f_i $ is the sensitivity of price to the $i$th factor times a unit change in the ith factor. This is sometimes called a ``partial duration.''
%			\item When the factors of interest are rates along the yield curve, the resulting partial durations are called ``key rate durations.''			
%		\end{itemize}
%	\end{itemize}
%	
%\end{frame}


\begin{frame}
	\frametitle{Convexity}
	
	\begin{itemize} \itemsep15pt
		\item Convexity is a measure of the curvature of the price-rate curve.
		
		\item It is used to improve upon duration-based approximations and hedging strategies
		
		\item A long position in non-callable bonds always has positive convexity.
		
		\item All else being equal, positive convexity is a desirable property for a long position; negative convexity is a bad thing 
		
		\begin{itemize} \itemsep10pt \vspace{10pt}
			\item Positive convexity means that duration underestimates the price increase resulting from a
			drop in yields, and overestimates the price decrease from an increase in yields.			
		\end{itemize}
	\end{itemize}
	
\end{frame}



\begin{frame}
	\frametitle{Convexity}
	
	
	\begin{itemize} \itemsep10pt
		\item Convexity is found by taking the second derivative of the bond price function and then dividing by the price $= \frac{1}{P} \frac{d^2 P}{d y^2}$ for an option-free bond:
		\begin{itemize}
			\item $C_{Mod} =\sum_{i=1}^{N} t_i^2  \frac{ K_i  e^{-y_c t_i}}{P}$
			
			\item $C_{Mod} = \sum_{i=1}^{N} t_i \left( t_i + \frac{1}{m} \right)\frac{K_i \left( 1+ y_m/m \right)^{-(mt_i+2)}}{P} $ \vspace{10pt}
			
			
			\item $C_{Mac} = C_{Mod} \times (1+y/m)^2	$
	
			\item Note that convexity is measured in terms of ``years squared.'' The units 	have no intuitive interpretation.
			\item Dollar convexity: $C \times P$
		\end{itemize}
		
		
	\end{itemize}
	
\end{frame}




\begin{frame}
	\frametitle{Using Convexity to Improve Price Sensitivity Estimates}
	
	\begin{block}{Taylor series}
		$$
		f(x)-f(x_0) \approx f'(x_0)(x-x_0) + \frac{1}{2}f''(x_0)(x-x_0)^2
		$$
		$$
		\Rightarrow P(y)-P(y_0) \approx P'(y_0)(y-y_0) + \frac{1}{2}P''(y_0)(y-y_0)^2
		$$
	\end{block}
	
	\begin{itemize} \itemsep10pt
		\item $P'(y_0)$: $-$Dollar duration, $P''(y_0)$: Dollar convexity
		$$
		\Rightarrow \frac{P(y)-P(y_0)}{P(y)}  \approx \frac{P'(y_0)}{P(y)}(y-y_0) + \frac{1}{2}\frac{P''(y_0)}{P(y)}(y-y_0)^2
		$$
		\item Percent change in price:
		$$
		\frac{\Delta P}{P} \approx -D \Delta y + \frac{1}{2} C \Delta y^2 
		$$
	\end{itemize}
	
\end{frame}




\begin{frame}
	\frametitle{Hedging with Duration and Convexity}
	
	
	\begin{itemize} \itemsep15pt
		\item Change in bond price with a change in
		interest rates:
		$$
		\Delta P \approx -D P \Delta y + \frac{1}{2} C P \Delta y^2 
		$$
		
		\item A delta neutral portfolio equates the hedge ratio (duration) of assets and liabilities.
		
		\item A gamma neutral portfolio is delta neutral, and also
		equates the gammas (convexity) of assets and liabilities.
		
	\end{itemize}
	
\end{frame}



\begin{frame}
	\frametitle{Example}
	
	
	\begin{itemize} \itemsep15pt
		\item A dealer in corporate bonds finds herself with an inventory of \$1mm in a 5 year 6.9\% bonds (semiannual payments) at the end of the trading day, priced at par.
		\item The bonds are illiquid, so selling them would entail a loss. Holding them overnight is risky, since their price might fall if rates rise.
		\item An alternative to selling the corporate bonds is to short more liquid Treasury bonds.  The following bonds are available:
		
		\begin{itemize}
			\item 10 yr, 8\% Treasury, $p = \$1,109.0$ per \$1,000 face
			\item 3 yr, 6.3\% Treasury, $p = \$1,008.1$ per \$1,000 face
		\end{itemize}
		
		\vspace{15pt}
		\begin{enumerate}[a]
			\item How much of the 10 year bond would she need to short to hedge? How much of the 3 year bond?
			\item If yields rise by 1\% overnight on all the bonds, show the result of the transactions the next
			day when the short position is closed out.
		\end{enumerate}
	\end{itemize}
	
\end{frame}



\begin{frame}
	\frametitle{Example (cont'd)}
	
	\begin{enumerate} \itemsep10pt
		\item Find modified duration of the bond to be hedged
		\begin{itemize}
			\item For 5 year 6.9\% bond: $y = 6.9\%, D=4.1685$
		\end{itemize}
		
		\item Find modified duration of the bonds to be shorted (find yields first)
		\begin{itemize}
			\item For 10 year 8\% bond: $y=6.5\%, D=7.004$
			\item For 3 year 6.3\% bond: $y = 6.00\%, D = 2.6999$
		\end{itemize}
		
		\item Find $x$ and $y$ in:
		
		\begin{itemize}
			\item $\$1m(4.1688)+x(7.005)=0  \Rightarrow x =\$595,167 $
			\item $\$1m(4.1688)+y(2.7)=0 \Rightarrow y = \$1,543,947 $
		\end{itemize}
		
		\item If yields rise by 1\% overnight on all the bonds:
		\begin{itemize}
			\item For 5 yr, yield to 7.9\%,: $P=\$959.423/1,000=0.959423$, Loss: $(\$1,000,000)(1-0.959423) = \$40,577 $
			\item For 10 year yield to 7.5\% $P=\$1034.81/1109=0.93311$, Loss: $(\$595,167)(1-0.93311) = \$39,810$
			\item For 3 year yield to 7\%: $P=\$981.42/\$1,008.1=0.97353$, Loss: $(\$1,543,947)(1-.97353) = \$40,861 $
		\end{itemize}
		
	\end{enumerate}
	
\end{frame}




\begin{frame}
	\frametitle{Example (cont'd)}
	
	\begin{itemize} \itemsep10pt
		\item What if the dealer wants the added protection of doing a gamma neutral hedge?
		
		\item Investment must be both delta neutral and gamma neutral. This requires matching deltas and gammas, and requires
		investments in both bonds.
		\begin{itemize}
			\item $P1 = \$1 m, D1 = 4.1685, C1 = 21.036$
			\item $P2 = ?, D2 = 7.004, C2 = 62.98$
			\item $P3 = ?, D3 = 2.699, C3 = 8.939$
		\end{itemize}
		
		\item Match hedge ratios:
		\begin{itemize}
			\item $ \$1m(4.1685) = P2(7.004) + P3(2.699)$
		\end{itemize}
		
		\item Match gamma:
		\begin{itemize}
			\item $ \$1m(21.038) = P2(62.98) + P3(8.939) $
		\end{itemize}
		
		\item 2 linear equations in two unknowns. Solve for P2 and P3.
		
	\end{itemize}
	
\end{frame}


\end{document} 